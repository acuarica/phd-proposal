
\chapter{Introduction}

In programming language design, the main goal of a \emph{static} type system is to prevent certain kind of errors from happening at run-time.
A type system is formulated as a set of constraints that gives any expression or term in a program a well-defined type.
As~\cite{pierceTypesProgrammingLanguages2002} states: ``A type system can be regarded as calculating a kind of \emph{static} approximation to the run-time behaviors of the terms in a program.''
These constraints are enforced by the \emph{type-checker} either when compiling or linking the program.
Thus, any program not satisfying the constraints stated within a type system is simply rejected by the type-checker.

Nevertheless, often the static approximation provided by a type system is not precise enough.
Being static, the analysis done by the type-checker needs to be conservative:
It is
% REVIEW: underline
\cancel{preferably}
to reject programs that are valid, but their validity cannot be ensured by the type-checker,
rather than accept some invalid programs.
However, there are situations when the developer has more information about the program that is too complex to explain in terms of the typing constraints.
To that end, programming languages often provide \emph{mechanisms} that 
make the typing constraints less strict
% the constraints of a type system can be relaxed
to permit more programs to be valid,
at the expense of causing more errors at run-time.
% Within a type system,
These mechanisms are essentially two:
\emph{Unsafe Intrinsics} and \emph{Casting}.

\textbf{Unsafe Intrinsics.}
Unsafe intrinsics is the ability to perform certain operations \emph{without} being checked by the compiler.
They are \emph{unsafe} because any misuse made by the programmer can compromise the entire system, \eg{},
corrupting data structures without notice, or
crashing the run-time system.
Unsafe intrinsics can be seen in safe languages, \eg{},
\lang{Java},
\lang{C\#},
\lang{Rust}, or
\lang{Haskell}.
Foreign Function Interface (\emph{FFI}), \ie{}, calling native code from within a safe environment is unsafe.
It is so because the run-time system cannot guarantee that the native code will not compromise the entire system.
In addition to FFI, some safe languages offer so-called \emph{unsafe} blocks, \ie{}, making unsafe operations within the language itself, \eg{},
\lang{C\#}\footnote{\url{https://docs.microsoft.com/en-us/dotnet/csharp/language-reference/language-specification/unsafe-code}}
and
\lang{Rust}\footnote{\url{https://doc.rust-lang.org/book/second-edition/ch19-01-unsafe-rust.html}}
.
Other languages provide an API to perform unsafe operations, \eg{},
\lang{Haskell}\footnote{\url{http://hackage.haskell.org/package/base-4.11.1.0/docs/System-IO-Unsafe.html}}
and
\lang{Java}.
But in the case of \lang{Java}, the API to make unsafe operations,
\code{sun.misc.Unsafe},
is unsupported\footnote{\url{http://www.oracle.com/technetwork/java/faq-sun-packages-142232.html}}
and undocumented.
It was originally intended for internal use within the JDK, but as we shall see later on, it is used outside the JDK as well.

\textbf{Casting.}
Programming languages with subtyping such as \java{} or \cpp{} provide a mechanism to \emph{view} an expression as a different type as it was defined.
This mechanism is often called \emph{casting} and takes the form \code{(T)t}.
Casting can be in two directions: \emph{upcast} and \emph{downcast}.
An upcast conversion happens when converting from a reference type $S$ to a reference type $T$, provided that $T$ is a \emph{supertype} of $S$.
An upcast does not require any explicit casting operation nor compiler check.
However, as we shall see later on, there are situations where an upcast requires an explicit casting operation.
On the other hand, a downcast happens when converting from a reference type $S$ to a reference type $T$, provided that $T$ is a \emph{subtype} of $S$.
Unlike upcasts, downcasts require a run-time check to verify that the conversion is indeed valid.
This implies that downcasts provide the means to bypass the static type system.
By avoiding the type system, downcasts can pose potential threats, because it is like the developer saying to the compiler: \emph{``Trust me here, I know what I'm doing''}.
Being a escape-hatch to the type system, a cast is often seen as a design flaw or code smell~\cite{tufanoWhenWhyYour2015} in an object-oriented system.


\section{Research Question}

If static type systems aim to prevent certain kind of errors from happening at run-time,
yet they provide the means to circumvent their constraints,
why exactly does one need to do so?
Are these mechanisms actually used in real-world code?
If yes, then how so?
This triggers our \textbf{main research question}:

% \rquestion{Are there \emph{unexpected usages of language features} in-the-wild that can give new insights to language designers, tools builders, researchers and developers?}
% \rquestion{How do developers circumvent the static type system's constraints?}
% \rquestion{What do developers circumvent the static type system's constraints for?}
% DONE: New research question
\rquestion{For what purpose do developers circumvent static type systems?}

% DONE: Remove this sentence on Java.
% We want to understand to what degree \java{}'s type system is useful.
% TODO: ADD examples with other languages
We have confidence that this knowledge can be:
\begin{inparaenum}[a)]
\item useful to make informed decisions for current \& future language designers,
% not only \java{},
\eg{}, the adoption of \emph{Variable Handles} in \java{} 9~\cite{jep193}
\item a reference for tool builders, \eg{}, by providing more precise or new refactoring analyses,
\item a guide for researchers to test new language features, \eg{}, \cite{wintherGuardedTypePromotion2011} or to carry out controlled experiments about programming, \eg{}, \cite{stuchlikStaticVsDynamic2011} and
\item a guide for developers for best or better practices.
\end{inparaenum}

% DONE: Change section "Proposal Plan" -> "Plan"
\section{Plan}

To answer our question above,
we plan to empirically study how the two aforementioned mechanisms
--- unsafe intrinsics and casting ---
are used by developers.
Since any kind of language study must be language-specific,
our plan is to focus on \java{} given its wide usage and relevance for both
% DONE: Cite Tiobe ,... (PS / PF2 slides)
research and industry.\footnote{\url{https://www.tiobe.com/tiobe-index/}}
Moreover, we focus on the \java{} Unsafe API to study unsafe intrinsics,
given than the Java Native Interface already has been studied in~\cite{tanSafeJavaNative2006,tanEmpiricalSecurityStudy2008,kondohFindingBugsJava2008,sunNativeGuardProtectingAndroid2014,liFindingBugsExceptional2009}.
In Chapter~\ref{cha:literature-review} we give a review of the literature in empirical studies of programming languages features.
Sections~\ref{sec:lr:unsafe} and~\ref{sec:lr:casting} reviews the \emph{state-of-the-art} of the different aspects related to the two proposed studies.

To better drive our \emph{main research question},
we propose to answer the following set of research sub-questions.
To answer these research sub-questions,
we already have devised (for the Unsafe \api{}) and
we plan to devise (for casting)
% REVIEW: underline
\underline{usage patterns}.
We believe that having usage patterns can help us to better categorize use cases and thus understanding how those mechanisms are used.
These patterns can provide an insight on how the language is being used by developers in real-world applications.
Overall these sub-questions will help us to answer our MRQ:

% TODO: Make this subsection.
\textbf{\todo{Make this subsection} Unsafe API.}
\begin{enumerate}[label=$URQ\arabic*:$,ref=$URQ\arabic*$,leftmargin=3.4\parindent]
\item\label{unsafe:rq1}
{\bf Does Unsafe impact common application code?}
We want to understand to what extent third-party code actually uses Unsafe.

% REVIEW: Why underlined
\item\label{unsafe:rq2}
{\bf \underline{Why} are Unsafe features used?}
We want to investigate what functionality third-party libraries require from Unsafe.
This could point out ways in which the \java{} language and/or the JVM need to be evolved to provide the same functionality, but in a safer way.
\end{enumerate}

These questions have been already answered in our previous published study on the Unsafe API in \java{}~\cite{mastrangeloUseYourOwn2015}.
In Chapter~\ref{cha:unsafe} we present a summary of this study.

% TODO: Make this subsection.
\textbf{\todo{Make this subsection} Casting.}
\begin{enumerate}[label=$CRQ\arabic*:$,ref=$CRQ\arabic*$,leftmargin=3.4\parindent]
\item\label{casts:rq1}
{\bf Is casting used in common application code?}
We want to understand to what extent application code actually uses casting operations.

\item\label{casts:rq2}
{\bf How and \underline{why} casts are used?}
If casts are actually used in application code,
we want to know how and why developers need to escape the type system.

\item\label{casts:rq3}
{\bf How recurrent are the reasons for which casts are used?}
In addition to understand how and why casts are used,
we want to measure how often developers need to resort to certain idioms to solve a particular problem.
\end{enumerate}

Finally, in Chapter \ref{cha:casts} we present our
% TODO: strikeout proposal
\cancel{proposal} plan for the \emph{casting} study, showing the results we have so far.

% The rest of this proposal is organized as follows:
