
\chapter{Introduction}

In programming language design, the main goal of a \emph{static} type system is to prevent certain kind of errors from happening at run-time.
A type system is formulated as a set of constraints that gives any expression or term in a program a well-defined type.
As~\cite{pierceTypesProgrammingLanguages2002} states: ``A type system can be regarded as calculating a kind of \emph{static} approximation to the run-time behaviors of the terms in a program.''
These constraints are enforced by the \emph{type-checker} either when compiling or linking the program.
Thus, any program not satisfying the constraints stated within a type system is simply rejected by the type-checker.

Nevertheless, often the static approximation provided by a type system is not expressive enough.
This situation arises when the developer has more information about the program that is too complex to explain in terms of the typing constraints.
In other words, the analysis done by the type-checker can not detect this kind of situation.
This is why programming languages usually offer mechanisms to escape the constraints imposed by a type system.
These mechanisms come essentially in two forms: \emph{Unsafe code} and \emph{casting}.

Unsafe code is the ability to bypass any kind of run-time check created by the compiler.
It is a characteristic of safe languages, \eg{}, \lang{Java}, \lang{C\#} or \lang{Haskell}.
Foreign Function Interface (\emph{FFI}), \ie{}, calling native code from within a safe environment is unsafe.
It is so because the run-time system cannot guarantee that the native code will not compromise the entire system.
In addition to FFI, safe languages offers the so-called \emph{unsafe} blocks, \ie{}, making unsafe operations within the language itself.
\lang{C\#}\footnote{\url{https://docs.microsoft.com/en-us/dotnet/csharp/language-reference/language-specification/unsafe-code}}
and \lang{Haskell}\footnote{\url{http://hackage.haskell.org/package/base-4.11.1.0/docs/System-IO-Unsafe.html}}
provide this kind of escape-hatch.
The case of \lang{Java} is a bit different.
\lang{Java} has an API to make unsafe operations --- the \code{sun.misc.Unsafe} class --- 
but it is undocumented.\footnote{\url{http://www.oracle.com/technetwork/java/faq-sun-packages-142232.html}}

% Casting on the other hand is the mechanism that allows a developer to assert that a certain expression has a certain run-time type.

% What can go wrong with type systems?




\section{Research Question}

We want to understand to what degree \java{}'s type system is useful.
We propose two studies to help understand these issues
How unsafe is isued
How cast is being used


Understanding how language features are used can give many insights to language designers, tools builders, researchers and developers.
This triggers our research question:

\rquestion{Are there \emph{unexpected usages of language features} in-the-wild that can give new insights to language designers, tools builders, researchers and developers?}

We believe that we --- as a research community --- should understand what kinds of programs are written in real code-bases.
We can use this information to improve several aspects of the software development process and supporting informed decisions for the driving forces mentioned above.
This fact opens the door for empirical studies about language features and their use in source code repositories, \eg{},
\github\footnote{\url{https://github.com/}} or
\gitlab\footnote{\url{https://gitlab.com/}},
and package managers repositories, like
\mavencentral\footnote{\url{http:/central.sonatype.org/}}.
Since any kind of language study must be language-specific,
our plan is to focus on \java{} given its wide usage and relevance for both research and industry.

Our plan is to make to empirical studies on how these features are used by developers.
In this proposal, we plan to target two specific \java{} features, namely, casting and the unsafe \api{}.
We have devised --- for the unsafe \api{} --- and we plan to devise language and \api{} usage patterns.
We believe that having usage patterns can help us to better categorize features and thus understanding how the feature is actually used.

\begin{table}[htbp]
\caption{\emph{Per} Feature Research Question}
\centering
\begin{tabular}{ll}
\hline
Feature     & Research Question\\
\hline
Casting     & Why Developers Need to Circumvent \java{}'s Type System?\\
Unsafe API  & Is \java{} Safe?\\
\hline
\end{tabular}
\end{table}


\section{Proposal Outline}

The rest of this proposal is organized as follows:
Chapter \ref{cha:literature-review} gives a review of the literature in the \emph{state-of-the-art} of the different aspects related to our goal.
The following two chapters introduce our proposal plan for the selected features:
Chapter \ref{cha:casts} presents our \emph{casting} study.
Finally, Chapter \ref{cha:unsafe} shows the study we already made on the Unsafe \api{} in \java{}.

While the literature review gives a broad overview in the field, each of the following chapters have their own ``Related Work'' section. 
The rationale behind this organization is that we prefer to show how we improve over the \emph{state-of-the-art} after having presented our plan for each feature.
