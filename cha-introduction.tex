
\chapter{Introduction}


In programming language design, the main goal of a \emph{static} type system is to prevent certain kind of errors from happening at run-time.
A type system is formulated as a set of constraints that gives any expression or term in a program a well-defined type.
As~\cite{pierceTypesProgrammingLanguages2002} states: ``A type system can be regarded as calculating a kind of \emph{static} approximation to the run-time behaviors of the terms in a program.''
These constraints are enforced by the \emph{type-checker} either when compiling or linking the program.
Thus, any program not satisfying the constraints stated within a type system is simply rejected by the type-checker.

Nevertheless, often the static approximation provided by a type system is not expressive enough.
This situation arises when the developer has more information about the program that is too complex to explain in terms of the typing constraints.
In other words, the analysis done by the type-checker can not detect this kind of situation.
This is why programming languages usually offer mechanisms to escape the constraints imposed by a type system.
These mechanisms come essentially in two forms: \emph{Unsafe code} and \emph{casting}.

Unsafe code is the ability to bypass any kind of run-time check created by the compiler.
It is a characteristic of safe languages, \eg{}, \lang{Java}, \lang{C\#} or \lang{Haskell}.
Foreign Function Interface (\emph{FFI}), \ie{}, calling native code from within a safe environment is unsafe.
It is so because the run-time system cannot guarantee that the native code will not compromise the entire system.
In addition to FFI, safe languages offers the so-called \emph{unsafe} blocks, \ie{}, making unsafe operations within the language itself.
\lang{C\#}\footnote{\url{https://docs.microsoft.com/en-us/dotnet/csharp/language-reference/language-specification/unsafe-code}}
and \lang{Haskell}\footnote{\url{http://hackage.haskell.org/package/base-4.11.1.0/docs/System-IO-Unsafe.html}}
provide this kind of escape-hatch.
The case of \lang{Java} is a bit different.
\lang{Java} has an API to make unsafe operations --- the \code{sun.misc.Unsafe} class --- 
but it is undocumented.\footnote{\url{http://www.oracle.com/technetwork/java/faq-sun-packages-142232.html}}

Casting on the other hand is the mechanism that allows a developer to assert that a certain expression has a certain run-time type.

What can go wrong with type systems?




\section{Research Question}

We want to understand to what degree \java{}'s type system is useful.
This leads to our research question:

\rquestion{Why Developers Need to Circumvent \java{}'s? Type System}

We propose two studies to help understand these issues

How unsafe is isued
How cast is being used

\section{Proposal Outline}

Our plan is to make to empirical studies on how these features are used by developers.

\begin{itemize}
\item Chapter \ref{cha:literature-review} gives an overview of the literature.
\item Chapter \ref{cha:casts} presents what are the current issues.
\item Chapter \ref{cha:unsafe} introduces our plan to answer the aforementioned research questions.
\end{itemize}
