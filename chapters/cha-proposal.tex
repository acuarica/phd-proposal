
\chapter{Empirical Usage Patterns}
\label{cha:proposal}

% Discovering Unexpected Language Features Usages at Large-Scale by Empirical-based Patterns

\section{Casts Usage Patterns}

\section{The \java{} Unsafe \api{} in the Wild}

The material in this section is partially based on our previously published work on the Unsafe \api{}~\cite{mastrangeloUseYourOwn2015}.

Our study on unsafe we have devised several usage patterns.
Java is a safe language.
Its runtime environment provides strong safety guarantees that any Java application can rely on.
Or so we think.
We show that the runtime actually does not provide these guarantees for a large fraction of today's \java{} code.
Unbeknownst to many application developers, the Java runtime includes a "backdoor" that allows expert library and framework developers to circumvent Java's safety guarantees.
This backdoor is there by design, and is well known to experts, as it enables them to write high-performance "systems-level" code in \java{}.

For our study on \smu{}, we needed to discover usage patterns.
Given its a singleton class, we have collected call sites, and proceed with a semi-automatic analysis.
On the other hand, our study related to casts involved a much more complex analysis.
Therefore we have decided to implement it with manual inspection.

The exceptions mechanism is orthogonal to the features we target in this proposal.
For instance, we have detected a \smu{} pattern to \throw{} undeclared exceptions.
Similarly, closely related to /casting/, \cce{} is thrown when a cast is invalid.
Therefore, we believe that these kind of studies can be complementary for our research.
They can help us to understand how programmers handle exceptions in these scenarios.

For our study on \smu{}, we first tried using \boa{} with \sourceforge{}.
We found out that only few projects were using \smu{}.
In contrast, our final study using \maven{} found that an order of magnitude more were using \smu{}.

Given 
We want to recast our 



There is an extensive literature on how reflection is used in \java{}.
But casting was never studied in its own.
In this paper we have presented several cast patterns.
We expect this paper gives a clear understanding of how cast is used in \java{}.
We hope these patterns can aid language and tool designers,
researchers, and developers to take advantage of these essential feature.