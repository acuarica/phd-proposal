
\chapter{Literature Review}
\label{cha:literature-review}

% asdf
% \ql{}~\cite{schafer_algebraic_2017}
% \ql{}~\cite{schafer_algebraic_2017}
% \cite{avgustinov_ql:_2016}

% Adding new features to a language should --- /in theory/ --- increase programmers productivity.
% But once a language feature is released, little is known about how it is actually used by the developer community.
% Therefore, it is extremely difficult to assess how features in a programming language impact on programmers productivity.
% We argue that this information is of great value, because can give many insights to drive the future of any programming language.

% On the other hand, cite:hanenberg_faith_2010,hanenberg_why_2014 argue that human behavior, \ie{}, controlled experiments, should be applied to programming language usage and design.
% With this approach, it should be possible --- in principle --- to understand to what degree a language feature impacts on programming productivity.
% However, for any kind of controlled experiment to be valid, it must reflect reality.
% Otherwise, any conjecture derived from a controlled experiment can be considered truthful but useless.

% Finally, understanding what developers write is not only useful in the field of language design and controlled experiments.
% For instance, cite:livshits_defense_2015 argue that most software analysis tools exclude certain dynamic features, \eg{}, reflection, \setjmp{}/\longjmp{}, \jni [fn::https://docs.oracle.com/javase/8/docs/technotes/guides/jni/spec/jniTOC.html], \eval{}, \etc{}, from their analyses.
% They claim that in order to understand how the limits of analysis tools impact software, we also need to understand what kind of code is being written in the real world.


% This kind of cast is called \emph{semi guarded} casts~\cite{winther_guarded_2011}.

% * Relate null as theoretical point of view in the TAPL book.
% * Featherweight Java is sound unless you use cast, as a motivation
% provide a bridge between compile-time and runtime checking. 

% Notice that 
% and not on bytecode.
% We examine these kinds of artifacts to analyze how they use casting operations. 
% We use a bytecode analysis library to search for cast operations.
% \& lgtm, since it is quite powerful and efficient.
% However, there are some issues that limit us to perform a full study using your platform.
% In particular, the issues that we found are:
% This could point out ways in which the \java{} language need to be evolved to provide the same functionality, but in a safer way.

Understanding how language features and \api{}s are being used is a
broad topic.
There is plenty of research in computer science literature about
empirical studies of programs; which involves several directions directly or indirectly related to our topic.
Along the last decades, researchers always has been interested in
understanding what kind of programs programmers write.
The motivation behind these studies is quite broad and ---
together with the evolution of computer science itself ---
has shifted to the needs of researchers.

Already~\cite{knuth_empirical_1971} started to study \fortran{} programs.
By knowing what kind of programs arise in practice, a compiler optimizer can focus in those cases, and therefore can be more effective.
Alternatively, to measure the advantages between compilation and interpretation in \basic{}, \cite{hammond_basic_1977} has studied a representative dataset of programs.
Adding to Knuth's work, \cite{shen_empirical_1990} made an empirical study for parallelizing compilers.
Similar works have been done for \cobol{}~\cite{salvadori_static_1975,chevance_static_1978}, \pascal{}~\cite{cook_contextual_1982}, and \apl{}~\cite{saal_properties_1975,saal_empirical_1977} programs. 

The organization of this section is as follows: 
compilers writers. 
How benchmarks and corpuses relate to this kind of studies is
presented in \S\ref{sec:rw:benchmarks}. 
\S\ref{sec:rw:largescale} gives an overview of other large-scale studies either in \java{} or in other languages. 
Related to our cast study, in \S\ref{sec:rw:experiments} we show studies on how static type systems impact on programmers productivity. 
Code Patterns discovery is presented in \S\ref{sec:rw:patterns}. 
Finally, \S\ref{sec:rw:mining} gives an overview of what tools are available to extract information from a software repository,
while \S\ref{sec:rw:selection} of how to select good candidates projects. 

\section{Benchmarks and Corpora}
\label{sec:rw:benchmarks}

Benchmarks are crucial to properly evaluate and measure product development.
This is key for both research and industry.
One popular benchmark suite for \java{} is DaCapo~\cite{blackburn_dacapo_2006}.
This suite has been already cited in more than thousand publications, showing how important is to have reliable benchmark suites.

Another suite was developed by in~\cite{tempero_qualitas_2010}.
They provide a corpus of curated open source systems to facilitate empirical studies on source code.

For any benchmark or corpus to be useful and reliable, it must faithfully represent real world code.
Therefore, we argue how important it is to make empirical studies about what programmers write.

\section{Large-scale Codebase Empirical Studies}
\label{sec:rw:largescale}

In the same direction to our plan, \cite{callau_how_2013} perform a study of the dynamic features of \smalltalk{}. 
Analogously, \cite{richards_analysis_2010,richards_eval_2011} made a similar study, but in this case targeting \javascript{}'s dynamic behavior and in particular the \code{eval} function. 
Also for \javascript{}, \cite{madsen_string_2014} analyzed how fields are accessed via strings, while~\cite{jang_empirical_2010} analyzed privacy violations. 
Similar empirical studies were done for \php{}~\cite{hills_empirical_2013,dahse_experience_2015,doyle_empirical_2011} and \swift{}~\cite{reboucas_empirical_2016}.  

Going one step forward, \cite{ray_large-scale_2017} studied the correlation between programming languages and defects. 
One important note is that they choose relevant project by popularity, measured \emph{stars} in \github{}. 
We argue that it is more important to analyse projects that are \emph{representative}, not \emph{popular}. 

For \java{}, \cite{dietrich_contracts_2017-1} made a study about how programmers use contracts in \mavencentral{}.
For their analysis\footnote{\url{https://bitbucket.org/jensdietrich/contractstudy}}, they have use JavaParser\footnote{\url{http://javaparser.org/}}.

\cite{landman_challenges_2017} have analyzed the relevance of static analysis tools with respect to reflection. 
They made an empirical study to check how often the reflection \api{} is used in real-world code. 
They argue, as we do, that controlled experiments on subjects need to be correlated with real-world use cases, \eg{}, \github{} or \mavencentral{}. 
\cite{winther_guarded_2011} ~have implemented a flow-sensitive analysis that allows to avoid manually casting once a guarded \code{instanceof} is provided. 
\cite{dietrich_broken_2014} have studied how changes in \api{} library impact in \java{} programs. 
Notice that they have used the Qualitas Corpus~\cite{tempero_qualitas_2010} mentioned above for their study. 

\subsection*{\textbf{Exceptions}}

\cite{kery_examining_2016,asaduzzaman_how_2016} focus on exceptions. 
They made empirical studies on how programmers handle exceptions in \java{} code. 
The work done by~\cite{nakshatri_analysis_2016} categorized them in patterns. 
Whether~\cite{coelho_unveiling_2015} used a more dynamic approach by analysing stack traces and code issues in \github{}. 

\subsection*{\textbf{Functional Programming Features}}

Programming language design has been always a hot topic in computer science literature.
It has been extensively studied in the past decades.
For instance, there is a trend in incorporating functional programming features into mainstream object-oriented languages, \eg, lambdas in \java{} 8\footnote{\url{https://docs.oracle.com/javase/specs/jls/se8/html/jls-15.html\#jls-15.27}}, \cpp{}11\footnote{\url{http://www.open-std.org/jtc1/sc22/wg21/docs/papers/2006/n1968.pdf}} and \cs{} 3.0\footnote{\url{https://msdn.microsoft.com/en-us/library/bb308966.aspx\#csharp3.0overview\_topic7}}; or parametric polymorphism --- \ie{}, generics --- in \java{} 5\footnote{\url{https://docs.oracle.com/javase/1.5.0/docs/guide/language/generics.html}}\(^{,}\)\footnote{\url{http://www.oracle.com/technetwork/java/javase/generics-tutorial-159168.pdf}}.

Looking at the aforementioned examples, \cite{mazinanian_understanding_2017} and \cite{uesbeck_empirical_2016} studied how developers use lambdas in \java{} and \cpp{} respectively; while \cite{parnin_java_2011,parnin_adoption_2013} did the same for generics in \java{}.
This kind of studies give an insight of the adoption of lambdas and generics; which can drive future direction for language designers and tool builders, while providing developers with best practices.




The inclusion of generics in \java{} is closely related to collections. 
\cite{parnin_java_2011,parnin_adoption_2013} studied how generics were adopted by \java{} developers. 
They found that the use of generics do not significantly reduce the number of type casts. 

\cite{costa_empirical_2017} have mined \github{} corpus to study the use and performance of collections, and how these usages can be improved. 
They have found out that in most cases there is an alternative usage that improves performance. 

\section{Controlled Experiments on Subjects}
\label{sec:rw:experiments}

There is an extensive literature \perse{} in controlled experiments on subjects to understand several aspects in programming, and programming languages. 
For instance, \cite{soloway_empirical_1984} tried to understand the how expert programmers face problem solving. 
\cite{budd_theoretical_1980} made a empirical study on how effective is mutation testing. 
\cite{prechelt_empirical_2000} compared how a given --- fixed --- task was implemented in several programming languages. 

\cite{latoza_developers_2010} realize that, in essence, programmers need to answer reachability questions to understand large codebases. 

Several authors~\cite{stuchlik_static_2011,mayer_empirical_2012,harlin_impact_2017} measure whether using a static-type system improves programmers productivity. 
They compare how a static and a dynamic type system impact on productivity. 
The common setting for these studies is to have a set of programming problems. 
Then, let a group of developers solve them in both a static and dynamic languages. 

For these kind of studies to reflect reality, the problems to be solved need to be representative of the real-world code. 
Having artificial problems may lead to invalid conclusions. 

The work by~\cite{wu_how_2017,wu_learning_2017} goes towards this direction. 
They have examined programs written by students to understand real debugging conditions. 
Their focus is on ill-typed programs written in \haskell{}. 
Unfortunately, these dataset does not correspond to real-world code. 
Our focus is to analyze code by experienced programmers. 

Therefore, it is important to study how casts are used in real-world code. 
Having a deep understanding of actual usage of casts can led to 
Informed decisions when designing these kind of experiments. 

\section{Code Patterns Discovery} \label{sec:rw:patterns}

\cite{posnett_thex:_2010} have extended \asm{}~\cite{bruneton_asm:_2002,kuleshov_using_2007} to implement symbolic execution and recognize call sites. 
However, this is only a meta-pattern detector, and not a pattern discovery. 
\cite{hu_dynamic_2008} used both dynamic and static analysis to discover design patterns, while \cite{arcelli_design_2008} used only dynamic. 

Trying to unify analysis and transformation tools~\cite{vinju_how_2006}, \cite{klint_rascal:_2009} built \rascal, a DSL that aims to bring them together.  

\section{Tools for Mining Software Repositories} \label{sec:rw:mining}

When talking about mining software repositories, we refer to extracting any kind of information from large-scale codebase repositories. 
Usually doing so requires several engineering but challenging tasks. 
The most common being downloading, storing, parsing, analyzing and properly extracting different kinds of artifacts. 
In this scenario, there are several tools that allows a researcher or developer to query information about software repositories. 

\cite{dyer_boa:_2013,dyer_declarative_2013} built \boa{}, both a domain-specific language and an online platform\footnote{\url{http://boa.cs.iastate.edu/}}. 
It is used to query software repositories on two popular hosting services, \github \footnote{\url{https://github.com/}} and \sourceforge \footnote{\url{https://sourceforge.net/}}. 
The same authors of \boa{} made a study on how new features in \java{} were adopted by developers~\cite{dyer_mining_2014}. 
This study is based \sourceforge{} data. 
The current problem with \sourceforge{} is that is outdated. 

To this end, \cite{gousios_ghtorent_2013} provides an offline mirror of \github{} that allows researchers to query any kind of that data. 
Later on, \cite{gousios_lean_2014} published the dataset construction process of \github{}. 

Similar to \boa{}, \lgtm \footnote{\url{https://lgtm.com/}} is a platform to query software projects properties. 
It works by querying repositories from \github{}. 
But it does not work at a large-scale, \ie{}, \lgtm{} allows the user to query just a few projects. 
Unlike \boa{}, \lgtm{} is based on \ql{}, an object-oriented domain-specific language to query recursive data structures~\cite{avgustinov_ql:_2016}. 

On top of \boa{}, \cite{tiwari_candoia:_2017} built \candoia \footnote{\url{http://candoia.github.io/}}. 
Although it is not a mining software repository \perse{}, it eases the creation of mining applications. 

Another tool to analyze large software repositories is presented in~\cite{brandauer_spencer:_2017}. 
In this case, the analysis is dynamic, based on program traces. 
At the time of this writing, the service\footnote{\url{http://www.spencer-t.racing/datasets}} was unavailable for testing. 

\sourcegraph{}\footnote{\url{https://sourcegraph.com}} is a tool that allows regular expression and diff searches.
It integrates with source repositories.

\section{Selecting Good Representatives} \label{sec:rw:selection}

Another dimension to consider when analyzing large codebases, is how relevant the repositories are. 
\cite{lopes_dejavu:_2017} made a study to measure code duplication in \github{}. 
They found out that much of the code there is actually duplicated. 
This raises a flag when consider which projects analyze when doing mining software repositories. 

\cite{nagappan_diversity_2013} have developed the Software Projects Sampling (SPS) tool. 
SPS tries to find a maximal set of projects based on representativeness and diversity. 
Diversity dimensions considered include total lines of code, project age, activity, and of the last 12 months, number of contributors, total code churn, and number of commits.

\cite{saied_visualization_2015} Visualization API/usage
patterns/mining.

\cite{kechagia_undocumented_2014} analyzed how undocumented and
unchecked exceptions cause most of the exceptions in
Android applications.
