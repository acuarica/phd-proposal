
\chapter{Introduction}

In programming language design, the \emph{main} goal of a static type system is to prevent certain kind of errors from happening at runtime.
Thus, a type system is formulated as a collections of constraints that gives any expression in a program a well defined type.
Any program not satisfying all constraints imposed by the type system is simply rejected (usually implemented into compilers or linkers).


Almost all main
But sometimes, 

\cite{milnerTheoryTypePolymorphism1978}

What can go wrong with type systems?

% Escaping the type 
% Overpassing 
% language safety
% Circumvent Java safety features
% type-safety and memory safety

A simple search on github on \code{ClassCastException}

\footnote{\url{https://github.com/search?l=Java&q=ClassCastException&type=Issues}}

\footnote{\url{https://github.com/search?l=Java&q=ClassCastException&type=Commits}}

At the time of this writing, \circa{} $150K$ commits messages were found.

\footnote{\url{https://github.com/jenkinsci/extra-columns-plugin/commit/02d10bd1fcbb2e656da9b1b4ec54208b0cc1cbb2}}

\footnote{\url{https://github.com/orientechnologies/orientdb/commit/ab33da5a489ca07146d9665f76b372cb9e852a7f}}

\footnote{\url{https://github.com/GoldenGnu/jeveassets/commit/5f4750bc8cfa7eed8ad01efd8add2cd2cc9bd831}}

\footnote{\url{https://github.com/ethereum/ethereumj/commit/224e65b9b4ddcb46198a6f8faf69edc65d34d382}}

\section{Research Question}

We want to understand to what degree \java{}'s type system is useful.

\rquestion{Why Developers Need to Circumvent \java{}'s? Type System}

We propose two studies to help understand these issues

How unsafe is isued
How cast is being used
% \subtitle{Why Developers Need to Circumvent \java{}'s Type System?}
% Discovering Unexpected Language Features Usages at Large-Scale by Empirical-based Patterns
% Empirical Usage Patterns Discovery 

\section{Proposal Outline}

\begin{itemize}
\item Chapter \ref{cha:background} presents what are the current issues.
\item Chapter \ref{cha:literature-review} gives an overview of the literature.
\item Chapter \ref{cha:proposal} introduces our plan to answer the aforementioned research questions.
\end{itemize}
