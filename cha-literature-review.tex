

% TODO: made a study -> conducted a study
% TODO: Avoid "note that", "notice that"
% TODO: Static vs. Dynamic analysis in the literature review
% Add discussion or static vs. dynamic analyses for these kinds of studies.
% Why static, why not dynamic? where static is OK, where is reasonable, feasible?

% Refactor this chapter into four subsections:
% Which projects: Repositories - what is out there
% Selection of projects: Corpora - what to analyze
% Tools - how to analyze (static/dynamic, BOA, QL, ...)
% What questions do developers ask? Questions -what questions to answer (especially where they get close to type systems)

\chapter{Literature Review}
\label{cha:literature-review}

Understanding how developers use language features
and \api{}s is a broad topic.
There is plenty of research in the computer science literature about
empirical studies of programs which involves multiple \emph{dimensions}
directly related to our plan.
Over the last decades,
researchers always have been interested in understanding what
kind of programs developers write.
The motivation behind these studies is quite broad,
and has been shifted to the needs of researchers,
together with the evolution of computer science itself.

% DONE: Basic, move earlier
% DONE: Nothing after 1990?
% DONE: At least add a statement here acknowledging all the MSR work.
For instance,~to measure the advantages between compilation and interpretation in \basic{},
\cite{hammondBASICEvaluationProcessing1977} studied a representative dataset of programs.
\cite{knuthEmpiricalStudyFORTRAN1971} started to study \fortran{} programs.
By knowing what kind of programs arise in practice,
a compiler optimizer can focus in those cases,
and therefore can be more effective.
Adding to Knuth's work,
\cite{shenEmpiricalStudyFortran1990} conducted an empirical study for
parallelizing compilers.
Similar works have been done for
\cobol{}~\cite{salvadoriStaticProfileCOBOL1975,chevanceStaticProfileDynamic1978},
\pascal{}~\cite{cookContextualAnalysisPascal1982},
and \apl{}~\cite{saalPropertiesAPLPrograms1975,saalEmpiricalStudyAPL1977} programs.
\cite{millerEmpiricalStudyReliability1990,millerFuzzRevisitedReexamination1995,forresterEmpiricalStudyRobustness2000}
studied the reliability of programs using \emph{fuzz} testing.
\cite{dieckmannStudyAllocationBehavior1999} studied the memory allocating
behavior in the SPECjvm98 benchmarks.%
\footnote{\url{https://www.spec.org/jvm98/}}
The importance of conducting empirical studies of programs
gave rise to the
International Conference on Mining Software Repositories%
\footnote{\url{http://www.msrconf.org/}}
in 2004.

% The organization of the rest of this chapter is as follows: 
When conducting empirical studies about programs,
multiple dimensions are involved.
The first question when conducting empirical stu of programs is 
what to analyze?
How benchmarks and corpora relate to this kind of study is presented in \S\ref{sec:lr:benchmarks}.
In \S\ref{sec:lr:largescale} we give an overview of other large-scale studies either in \java{} or in other languages.
Code Patterns discovery is presented in \S\ref{sec:lr:patterns}.
An overview of what tools are available to extract information from software repositories is given in \S\ref{sec:lr:mining},
while \S\ref{sec:lr:selection} shows how to select good candidates projects from a large-base software repository.
Finally in \S\ref{sec:lr:unsafe} and \S\ref{sec:lr:casting} we present the related work more specific to the Unsafe API and Casting respectively.


\section{Benchmarks and Corpora}
\label{sec:lr:benchmarks}

% TODO: OOPSLA'17 code duplicates in GitHub
% TODO: SPEC, etc
Benchmarks are crucial to properly evaluate and measure product development.
This is key for both research and industry.
One popular benchmark suite for \java{} is the DaCapo Benchmark~\citep{blackburnDaCapoBenchmarksJava2006}.
This suite has been already cited in more than thousand publications, showing how important is to have reliable benchmark suites.

Another suite has been developed by~\cite{temperoQualitasCorpusCurated2010}.
They provide a corpus of curated open source systems to facilitate empirical studies on source code.
On top of Qualitas Corpus,~\cite{dietrichXCorpusExecutableCorpus2017} provide an executable corpus of \java{} programs.
This allows any researcher to experiment with both static and dynamic analysis.

% TODO: Some discussion of the representativeness of the corpora
For any benchmark or corpus to be useful and reliable, it must faithfully represent real world code.
Along these lines, \cite{allamanisMiningSourceCode2013} go one step further and provide a large-scale (14,807) curated corpus of open source \java{} projects.
\todo{Acknowledge here that open source repositories, while not curated corpora, are very commonly used (might be a forward pointer, but necessary here)}

% DONE: Remove subsection
% \subsection*{Selecting Good Representatives}
% \label{sec:lr:selection}

% TODO: Related to Lopes
Another dimension to consider when analyzing large codebases, is how relevant the repositories are.
\cite{lopesDeJaVuMapCode2017} conducted a study to measure code duplication in \github{}.
They found out that much of the code there is actually duplicated.
This raises a flag when considering which projects to analyze when mining software repositories.

\cite{baxterCloneDetectionUsing1998} propose a clone detection algorithm using Abstract Syntax Trees,
\todo{How do clones fit in this section?}
while \cite{riegerVisualDetectionDuplicated} propose a visual detection for clones.
\cite{yuanCMCDCountMatrix2011,chenReplicationReproductionCode} instead propose Count Matrix-based approach to detect code clones.

% DONE: Move to 2.1
% DONE: Removed "of the last 12 months"
\cite{nagappanDiversitySoftwareEngineering2013} have developed the Software Projects Sampling (SPS) tool.
SPS tries to find a maximal set of projects based on representativeness and diversity.
Diversity dimensions considered include total lines of code,
project age, activity, number of contributors, total code churn,
and number of commits.

\section{Tools for Mining Software Repositories}
\label{sec:lr:mining}

% DONE: Too general, should be statistical information?
When talking about mining software repositories,
we refer to extracting any kind of information from large-scale codebase repositories. 
Usually doing so requires several engineering but challenging tasks.
The most common being downloading, storing, parsing, analyzing and properly extracting different kinds of artifacts.
In this scenario, there are several tools that allows a researcher or developer to query information about software repositories.

% DONE: Fix .QL citation
\cite{urmaProgrammingLanguageEvolution2012} evaluated seven source code
query languages\footnote{\url{https://wiki.openjdk.java.net/display/Compiler/Java+Corpus+Tools}}:
\emph{Java Tools Language} \citep{cohenJTLJavaTools},
\emph{Browse-By-Query}\footnote{\url{http://browsebyquery.sourceforge.net/}},
\emph{SOUL} \citep{derooverSOULToolSuite2011},
\emph{JQuery} \citep{volderJqueryGenericCode2006},
\emph{.QL} \citep{moorKeynoteAddressQL2007},
\emph{Jackpot}\footnote{\url{http://wiki.netbeans.org/Jackpot}}, and
\emph{PMD}\footnote{\url{https://pmd.github.io/}}.
They have implemented --- whenever possible --- four use cases using the tools mentioned above.
They concluded that only \emph{SOUL} and \emph{.QL} have the minimal features to implement all their use cases.

% DONE: Which features? Any relevant for your work?
\cite{dyerBoaLanguageInfrastructure2013,dyerDeclarativeVisitorsEase2013} built \boa{}, both a domain-specific language and an online platform\footnote{\url{http://boa.cs.iastate.edu/}}. 
It is used to query software repositories on two popular hosting services, \github{} and \sourceforge{}.
The same authors of \boa{} conducted a study on
how new \java{} features, \eg,
\emph{Assertions},
\emph{Enhanced-For Loop},
\emph{Extends Wildcard},
were adopted by developers over time~\citep{dyerMiningBillionsAST2014}.
This study is based \sourceforge{} data.
The current problem with \sourceforge{} is that is outdated.

To this end, \cite{gousiosGHTorentDatasetTool2013} provides an offline mirror of \github{} that allows researchers to query any kind of that data.
Later on, \cite{gousiosLeanGHTorrentGitHub2014} published the dataset construction process of \github{}.

Similar to \boa{}, \lgtm{}\footnote{\url{https://lgtm.com/}} is a platform to query software projects properties.
It works by querying repositories from \github{}.
But it does not work at a large-scale, \ie{}, \lgtm{} allows the user to query just a few projects.
Unlike \boa{}, \lgtm{} is based on \ql{}
--- before named \emph{.QL} ---,
% as mentioned in \S\ref{sec:lr:patterns} ---
an object-oriented domain-specific language to query recursive data structures~\cite{avgustinovQLObjectorientedQueries2016}.

% On top of \boa{},~\cite{tiwariCandoiaPlatformBuilding2017} built \candoia \footnote{\url{http://candoia.github.io/}}. 
% Although it is not a mining software repository \perse{}, it eases the creation of mining applications. 

Another tool to analyze large software repositories is presented in~\cite{brandauerSpencerInteractiveHeap2017}.
In this case, the analysis is dynamic, based on program traces. 
At the time of this writing, the service\footnote{\url{http://www.spencer-t.racing/datasets}} was unavailable for testing. 

\cite{bajracharyaSourcererInternetscaleSoftware2009} provide a tool to query large code bases by extracting the source code into a relational model.
Sourcegraph\footnote{\url{https://sourcegraph.com}} is a tool that allows regular expression and diff searches.
It integrates with source repositories to ease navigate software projects.

% DONE: Removed subsection Code Patterns Discovery: Body of this section seems to be on tools, not on pattern discovery
% \subsection*{Code Patterns Discovery}
% \label{sec:lr:patterns}

% DONE: What's a pattern?
% DONE: Why
\cite{posnettTHEXMiningMetapatterns2010} have extended
\asm{}~\citep{brunetonASMCodeManipulation2002,kuleshovUsingASMFramework2007}
to detect meta-patterns, \ie,
purely structural patterns of object-oriented interaction.
\cite{huDynamicAnalysisDesign2008} used both dynamic and static analysis to discover design patterns, while \cite{arcelliDesignPatternDetection2008} used only dynamic.

% DONE: Why is Rascal relevant?
Trying to unify analysis and transformation tools,
\cite{vinjuHowMakeBridge2006,klintRASCALDomainSpecific2009} built \rascal,
a DSL that aims to bring them together by querying the AST of a program.
% DONE: Redundant, remove
% As mentioned above,
% \cite{dietrichContractsWildStudy2017a} conducted a study about how programmers use contracts in \mavencentral{}.
% For their analysis\footnote{\url{https://bitbucket.org/jensdietrich/contractstudy}},

As its name suggests, JavaParser\footnote{\url{http://javaparser.org/}}
is a parser for \java.
The main issue with JavaParser is the lack to do symbol resolution integrated with the project dependencies.


\section{Large-scale Codebase Empirical Studies}
\label{sec:lr:largescale}

In the same direction as our plan, \cite{callauHowWhyDevelopers2013} performed an empirical study to assess how much the dynamic and reflective features of \smalltalk{} are actually used in practice.
Analogously, \cite{richardsAnalysisDynamicBehavior2010,richardsEvalThatMen2011} made a similar study, but in this case targeting \javascript{}'s dynamic behavior and in particular the \code{eval} function. 
Also for \javascript{}, \cite{madsenStringAnalysisDynamic2014} analyzed how fields are accessed via strings, while~\cite{jangEmpiricalStudyPrivacyviolating2010} analyzed privacy violations. 
Similar empirical studies were done for \php{}~\cite{hillsEmpiricalStudyPHP2013,dahseExperienceReportEmpirical2015,doyleEmpiricalStudyEvolution2011} and \swift{}~\cite{reboucasEmpiricalStudyUsage2016}.  

Going one step forward, \cite{rayLargescaleStudyProgramming2017} studied the correlation between programming languages and defects. 
One important note is that they choose relevant projects by popularity,
\underline{measured} \emph{stars} in \github{}.
\todo{Why is this important?}
% We argue that it is more important to analyse projects that are \emph{representative}, not \emph{popular}.

\cite{gorlaCheckingAppBehavior2014} mined a large set of Android applications, clustering applications by their description topics and identifying outliers in each cluster with respect to their API usage.
\cite{grechanikEmpiricalInvestigationLargescale2010} also mined large scale software repositories to obtain several statistics on how source code is actually written.

For \java{},~\cite{dietrichContractsWildStudy2017a} conducted a study
about how programmers use contracts in \mavencentral{}\footnote{\url{http://central.sonatype.org/}}.
\cite{dietrichBrokenPromisesEmpirical2014} have studied how
\api{} changes impact \java{} programs.
They have used the Qualitas Corpus~\citep{temperoQualitasCorpusCurated2010} mentioned above for their study.

\cite{tufanoWhenWhyYour2015,tufanoWhenWhyYour2017} studied when code smells are introduced in source code.
\cite{palombaLandfillOpenDataset2015}
contribute a dataset of five types of code smells together with a systematic procedure for validating code smell datasets.
\cite{palombaDetectingBadSmells2013} propose to detect code smells using change history information.

\subsection*{Exceptions}

\cite{keryExaminingProgrammerPractices2016,asaduzzamanHowDevelopersUse2016} focus on exceptions.
They conducted empirical studies on how programmers handle exceptions in \java{} code.
The work done by~\cite{nakshatriAnalysisExceptionHandling2016} categorized them into patterns.
\cite{coelhoUnveilingExceptionHandling2015} used a more dynamic approach by analysing stack traces and code issues in \github{}.

\cite{kechagiaUndocumentedUncheckedExceptions2014} analyzed how undocumented and
unchecked exceptions cause most of the exceptions in
Android applications.

\subsection*{Programming Language Features}

\todo{Instead of this, add an intro sentence motivating this section by connecting it to your focus on type safety (if possible).
Generics is not just a functional language feature, no?}
Programming language design has been always a hot topic in computer science literature.
It has been extensively studied in the past decades.
There is a trend in incorporating functional programming features into mainstream object-oriented languages, \eg,
lambdas in \java{} 8\footnote{\url{https://docs.oracle.com/javase/specs/jls/se8/html/jls-15.html\#jls-15.27}},
\cpp{}11\footnote{\url{http://www.open-std.org/jtc1/sc22/wg21/docs/papers/2006/n1968.pdf}} and
\cs{} 3.0\footnote{\url{https://msdn.microsoft.com/en-us/library/bb308966.aspx\#csharp3.0overview\_topic7}};
or \underline{parametric polymorphism} \todo{not strictly functional}  --- \ie{}, generics --- in \java{} 5\footnote{\url{https://docs.oracle.com/javase/1.5.0/docs/guide/language/generics.html}}\(^{,}\)\footnote{\url{http://www.oracle.com/technetwork/java/javase/generics-tutorial-159168.pdf}}.

\cite{mazinanianUnderstandingUseLambda2017} and \cite{uesbeckEmpiricalStudyImpact2016} studied how developers use lambdas in \java{} and \cpp{} respectively.
The inclusion of generics in \java{} is closely related to collections. 
\cite{parninJavaGenericsAdoption2011,parninAdoptionUseJava2013} studied how generics were adopted by \java{} developers.
They found that the use of generics do not significantly reduce the number of type casts.

\cite{costaEmpiricalStudyUsage2017} have mined \github{} corpus to study the use and performance of collections,
and how these usages can be improved.
They found that in most cases there is an alternative usage that
improves performance.

This kind of studies give an insight of the adoption of lambdas and generics; which can drive future direction for language designers and tool builders, while providing developers with best practices.

\subsection*{Unsafe API}
\label{sec:lr:unsafe}

% DONE: This paragraph should go first, no? 
Oracle provides the \smu{} class for low-level programming,
\eg, synchronization primitives, direct memory access methods,
array manipulation and memory usage.
Although the \smu{} class is not officially documented,
it is being used in industrial applications outside the JDK, compromising the safety of the \java{} ecosystem.

Oracle software engineer Paul Sandoz performed an informal analysis of
Maven artifacts and usages in Grepcode~\citep{sandoz-personal-communication}
and conducted a unscientific user survey to study how \unsafe{} is used~\citep{psandoz14}.
The survey consists of 7 questions\footnote{\url{http://www.infoq.com/news/2014/02/Unsafe-Survey}} 
that help to understand what pieces of \smu{} should be mainstreamed.
% DONE: Sounds wrong here. This wording fits in a paper, not a proposal.
In our work~\citep{mastrangeloUseYourOwn2015} we extend Sandoz' work
by performing a comprehensive study of the \mavencentral{}
software repository to analyze how and when \smu{} is being used.
This study is summarized in Chapter \ref{cha:unsafe}.

% DONE: Who cares that there is literature. Motivate the contents of that literature by putting a better glue sentence here..
% DONE: Relevant? These are users of Unsafe, not studies of it
% The use of \smu{} is not limited to industrial applications.
% Researchers have used \smu{} as well.
% \cite{korlandNoninvasiveConcurrencyJava2010} presented a \java{} STM framework, intended as a development platform for scalable concurrent applications and as a research tool for designing STM algorithms.
% They chose to use \smu{} to implement fast reflection,
% as it proved to be vastly more efficient than the standard \java{} reflection mechanisms.
% \cite{pukallFlexibleDynamicSoftware} introduced a runtime update approach based on Java that offers flexible dynamic software updates with minimal performance overhead.
% They used the \code{allocateInstance} method,
% because it eases the creation of instances even if the class has no default constructor.
% \cite{gligoricCoDeSeFastDeserialization2011} proposed a new approach to serialization/deserialization via code generation, using \smu{} to allocate instances and to set the fields.
% The Jikes RVM~\cite{alpernJikesResearchVirtual2005} is a Java Virtual Machine targeting researchers in runtime systems.
% It is a Java-in-Java virtual machine because is itself built in \java{}, a style of implementation termed meta-circular.
% The Jikes RVM provides an implementation of \smu{} with the \emph{magic} framework.
% \cite{framptonDemystifyingMagicHighlevel2009} proposed \code{org.vmmagic} to provide an escape hatch to low-level alternatives needed to build virtual machines; however, they require compiler support.

\cite{tanSafeJavaNative2006} propose a safe variant of \jni{}.
\cite{tanEmpiricalSecurityStudy2008,kondohFindingBugsJava2008} carried out an empirical security study to describe a taxonomy to classify bugs when using JNI.
\cite{sunNativeGuardProtectingAndroid2014} develop a method to isolate native components in Android applications.
\cite{liFindingBugsExceptional2009} analyze the discrepancy between how exceptions are handled in native code and \java{}.

\subsection*{Casting}
\label{sec:lr:casting}

\todo{Intro sentence needed}

% DONE: Makes it sound like "semantic patterns" was a well-known concept. Don't start with terms like this, start with description instead.
\cite{wintherGuardedTypePromotion2011} has implemented a
path sensitive analysis that allows the developer to avoid casting
once a guarded \code{instanceof} is provided.
He proposes four cast categorizations according to their
run-time type safety:
\emph{Guarded Casts}, \emph{Semi-Guarded Casts},
\emph{Unguarded Casts}, and \emph{Safe Casts}.
We plan to refine this categorization to answer
our~\ref{casts:rq2} (\emph{\crqB}).
This is described in Chapter~\ref{cha:casts}.

% DONE: Briefly connect to casts
\cite{tsantalisJDeodorantIdentificationRemoval2008} present an
Eclipse plug-in that identifies type-checking bad smells,
a "variation of an algorithm that should be executed,
depending on the value of an attribute".
They provide refactoring analysis to remove the detected smells
by introducing inheritance and polymorphism.
This refactoring will introduce casts to select
the right type of the object.

\cite{livshitsImprovingSoftwareSecurity2006,livshitsReflectionAnalysisJava2005} ``describes an approach to call graph construction for \java{} programs in the presence of reflection.''
He has devised some common usage patterns for reflection.
Most of the patterns use casts.
We plan to categorize all cast usages, not only where reflection is used.

\cite{landmanChallengesStaticAnalysis2017} have analyzed the relevance of static analysis tools with respect to reflection.
They made an empirical study to check how often the reflection \api{} is used in real-world code.
They have devised reflection AST patterns, which often involve the use of casts.
Finally, they argue that programming controlled experiments on subjects need to be correlated with real-world use cases, \eg{}, \github{} or \mavencentral{}.

% TODO: Shouldn't that 2.8?
% \subsection*{Controlled Experiments on Subjects}
% \label{sec:lr:experiments}
\textbf{Controlled Experiments on Subjects.}
There is an extensive literature \perse{} in controlled experiments on subjects to understand several aspects in programming, and programming languages.
For instance, \cite{solowayEmpiricalStudiesProgramming1984} tried to understand how expert programmers face problem solving.
\cite{buddTheoreticalEmpiricalStudies1980} made a empirical study on how effective is mutation testing.
\cite{precheltEmpiricalComparisonSeven2000} compared how a given --- fixed --- task was implemented in several programming languages.
%
\cite{latozaDevelopersAskReachability2010} realize that, in essence, programmers need to answer reachability questions to understand large codebases.
%
Several authors~\cite{stuchlikStaticVsDynamic2011,mayerEmpiricalStudyInfluence2012,harlinImpactUsingStaticType2017} measure whether using a static-type system improves programmers productivity.
They compare how a static and a dynamic type system impact on productivity.
The common setting for these studies is to have a set of programming problems.
Then, let a group of developers solve them in both a static and dynamic languages.
%
For these kind of studies to reflect reality, the problems to be solved need to be representative of the real-world code.
Having artificial problems may lead to invalid conclusions.
The work by~\cite{wuHowTypeErrors2017,wuLearningUserFriendly2017} goes towards this direction. 
They have examined programs written by students to understand real debugging conditions. 
Their focus is on ill-typed programs written in \haskell{}.

% Unfortunately, these dataset does not correspond to real-world code.
% Therefore, it is important to study how casts are used in real-world code. 
% This can led to informed decisions when designing these kind of experiments. 